\chapter{Régime sinusoïdal permanent}
\minitoc
\minilof
\minilot

\section{Réponse à une excitation sinusoïdale, régime transitoire, régime permanent}
    Soit un dipôle $R, C$ série par exemple, branché sur une source de tension fournissant une tension alternative sinusoïdale à partir de $t=0$. On a $e(t) = E\cos(\omega t + \varphi) Y(t)$ avec $Y$ la fonction de Heaviside. Initialement l'intisité est nulle et le condensateur est déchargé. L'équation différentielle vérifiée par $q$ la charge du condensateur est~: \[ \ddot{q} + \frac{R}{L} \dot{q} + \frac{q}{RC} = \frac{e}{L} \]. Pour $t$ postif~:
    \begin{itemize}
        \item La solution générale $q= f_T(t)$ de l'équation différentielle sans second membre correspond à un régime amorti, apériodique ou pseudo-périodique ;
        \item L'équation complète admet, comme on le démontreraplus loin, une solution particulière alternative sinusoïdale de pulsation égale à celle de la source de tension, de la forme $q = f_P(t) = Q \cos(\omega t +\psi)$.
        \item On a donc pour $t$ positif, $q = f_T(t) + f_P(t)$ et $i = \dot{q} = f_T' + f_P'$. Les constantes d'intégration étant déterminées par les conditions initiales~: la continuité de $i$ qui traverse l'inductance $L$ donne $\lim\limits{t \to 0} i = 0$ et celle de $q$ donne $\lim\limits{t \to 0} q = 0$. Pour $t$ suffisamment grand, le régime transitoire peut être considéré comme terminé, du fait des exponentielles d'argument négatifs, $\lim\limits{t \to \infty} f_T(t) = 0$. On a alors un régime sinusoïdal permanent~:
        \begin{align}
            q = f_P(t) &= Q \cos(\omega t +\psi) \\
            i = f_P'(t) &= -\omega Q\sin(\omega t +\psi)
        \end{align}
    \end{itemize}
\section{Déphasage entre deux grandeurs sinusoïdales de même fréquence}
    \subsection{Définition}
            Pour deux grandeurs sinusoïdale de même fréquence (de même pulsation)~: $x = X\cos(\omega t + \varphi)$ et $y = Y\cos(\omega t + \psi)$, ou $x = X\sin(\omega t + \varphi)$ et $y = Y\sin(\omega t + \psi)$, $X$ et $Y$ sont positifs, ce sont les amplitudes ; $\omega t + \varphi$ et $\omega t + \psi$ sont les phases à la date $t$. $\Phi = \psi -\varphi$ est le déphasage de $y$ par rapport à $x$ (ou différence de phase).
            Si l'on choisit l'origine des dates de telle façon qu'à $t=0$, $x=X$, on a alors $\varphi=0$ et $\Phi = \psi$.
    \subsection{Mesure du déphasage sur les graphes}
            Pour simplifier, on peut donc choisir l'origine des dates telle que $x = X\cos \omega t$ et $y=Y\cos(\omega t + \Psi)$ avec $\Phi \in \intervalleof{-\pi}{\pi}$. À $t' = t + \frac{\Phi}{\omega} = t + T\frac{\Phi}{2\pi}$, $x$ aura la phase qu'à $y$ à la date $t$. L'avance de $y$ sur $x$ est donc de $\theta = \frac{\Phi}{2\pi} T$.
            En éliminant le paramètre $t$ dans les expressions de $x$ et $y$~:
            \begin{align}
                \frac{x}{X} &= \cos\omega t \\
                \frac{y}{Y} &= \cos\omega t \cos\Phi -\sin\omega t \sin\Phi,
            \end{align}
            donc~:
            \begin{align}
                \cos\omega t &= \frac{x}{X} \\
                \sin\omega t &= \frac{\frac{x}{X}\cos\Phi - \frac{y}{Y}}{\sin\Phi}.
            \end{align}
            Avec $\sin^2+\cos^2=1$, on obtient~:
            \begin{align}
                1 &= \cos^2\omega t + \sin^2\omega t \\
                  &= \left(\frac{x}{X}\right)^2 + \left( \frac{\frac{x}{X}\cos\Phi - \frac{y}{Y}}{\sin\Phi} \right)^2 \\
                  &= \left(\frac{x}{X}\right)^2 + \frac{(\frac{x}{X}\cos\Phi - \frac{y}{Y})^2}{\sin^2\Phi} \\
       \sin^2\Phi &= \sin^2\phi \left(\frac{x}{X}\right)^2 + \cos^2\Phi \left(\frac{x}{X}\right)^2 + \left(\frac{y}{Y}\right)^2 - 2 \frac{xy}{XY} \cos\Phi
            \end{align}
            Donc on a bien l'équation cartésienne d'une ellipse~:
            \begin{equation}
                \left(\frac{x}{X}\right)^2 + \left(\frac{y}{Y}\right)^2 - 2 \frac{xy}{XY} \cos\Phi = \sin^2\Phi.
            \end{equation}
            Cette ellipse est inscrite dans le rectagle $[X, Y]$ et en général, ces axes sont inclinés par rapport au repère $(Oxy)$. Pour $y=0$, on a $x = \pm X\sin(\Phi) = \pm a$. Pour $x=0$ on a $y = \pm Y\sin\Phi$ donc $\abs{\sin\Phi} = \frac{2a}{2X} = \frac{2b}{2Y}$. Si l'ellipse est décrite dans le sens trigonométrique, $x$ est maximal avant $y$, $y$ est en retard sur $x$, alors $\Phi$ est négatif.
        \subsection{Déphasages particulier}
            Si $\Phi$ est nul, $x$ et $y$ sont en concordance de phase, ou "en phase". Alors $y=\frac{Y}{X} x$, le graphe de $y(x)$ est alors un segment de droite symétrique par rapport à $O$, de coefficient directeur positif. Si $\Phi=\pi$, $x$ et $y$ sont en opposition de phase. Alors $y=-\frac{Y}{X} x$, le graphe de $y(x)$ est alors un segment de droite symétrique par rapport à $O$, de coefficient directeur négatif. Si $\Phi=\frac{\pi}{2}$, $y$ est en quadrature avance sur $x$. Si $\Phi=-\frac{\pi}{2}$, $y$ est en quadrature retard sur $x$.

            Dans les deux derniers cas, $\cos\phi = 0$, $\sin\Phi=1$, l'équation de l'ellipse est $\left(\frac{x}{X}\right)^2 + \left(\frac{y}{Y}\right)^2 = 1$. Les axes de symétrie de l'ellipse sont $Ox$ et $Oy$. Si l'ellipse est décrite dans le sens trigonométrique, $y$ est en quadrature retard sur $x$ (on passe par $y$ après $x$). Si elle est décrite dans le sens inverse, $y$ est en quadrature avance sur $x$ (on passe par $y$ avant $x$).
\section{Notation complexe d'une grandeur sinusoïdale}
    \label{sec:complexe}
    \subsection{Grandeur complexe associée, amplitude complexe}
        En électricité, pour éviter toute confusion avec l'intensité, on note $\ju$ l'unité imaginaire $\ju^2=-1$, alors qu'en mécanique ou en optique, on note ce nombre $\iu$. Soit la grandeur $x$ fonction sinusoïdale de la date $t$~: $x = X\cos(\omega t +\varphi)$. La grandeur complexe associée à $x$ est par définition~: $\xc = X\exp(\ju(\omega t +\varphi))$. L'amplitude complexe de $x$ est $\Xc = X\exp(\ju \varphi)$, donc $\xc = \Xc \exp(\ju\omega t)$. On a donc $x = \Re(\Xc)$, $X = \abs{\Xc}$ et $\varphi = \Arg(\Xc)$.
    \subsection{Dérivation d'une grandeur complexe associée}
        Avec les grandeurs complexes associées, la dérivation par rapport à $t$ devient une multiplication par $\ju\omega$~:
        \begin{align}
            \derived{\xc}{t} &= \ju\omega\xc \\
            \derivedn{\xc}{t}{n} &= (\ju\omega)^n\xc.
        \end{align}
        Pour l'intégration, si à $t=0$ on a $x=x_0$, alors
        \begin{equation}
                \int \xc \diff t = \frac{\xc-\xc_0}{\ju\omega}.
        \end{equation}
    \subsection{Équations différentielles linéaires à coefficients constants}
        Une équation différentielle du type~:
        \begin{equation}
            \label{eq:eqdiffcomp}
            \yc = \sum_{i=1}^n \alpha_i \derivedn{\xc}{t}{i}
        \end{equation}
        peut s'écrire, après division de tous les termes par $\exp(\ju\omega t)$~:
        \begin{align}
            \Yc &= \Xc \sum_{i=1}^n \alpha_i (\ju \omega)^i\\
                          &= \Xc ~ \Zc
        \end{align}
        avec $\Zc = \sum_{i=1}^n \alpha_i (\ju \omega)^i$. La grandeur complexe $\xc = \Xc\exp(\ju\omega t)$ est solution de l'équation différentielle. Sa partie réelle $X\cos(\omega t+\varphi)$ est solution de l'équation différentielle réelle $\Re(\eqref{eq:eqdiffcomp})$~:
        \begin{equation}
            \label{eq:eqdiffreel}
            Y\cos(\omega t +\psi) = \sum_{i=1}^n \alpha_i \derivedn{x}{t}{i}.
        \end{equation}
        La réciproque est aussi vraie~: Si $x=X\cos(\omega t +\varphi)$ est une solution de l'équation réelle \eqref{eq:eqdiffreel}, alors $X\sin(\omega t +\varphi)$ est solution de l'équation~:
        \begin{equation}
            \label{eq:eqdiffreel2}
            Y\sin(\omega t +\psi) = \sum_{i=1}^n \alpha_i \derivedn{x}{t}{i},
        \end{equation}
        qui est identique à \eqref{eq:eqdiffreel} à un changement d'origine ds dates près ($T/4$).

        Mais $\eqref{eq:eqdiffcomp} = \eqref{eq:eqdiffreel} + \ju \eqref{eq:eqdiffreel2}$, donc $\Xc = \Xc \exp(\ju\omega t)$ est solution de l'équation complexe \eqref{eq:eqdiffcomp}. La recherche de la solution particulière sinusoïdale de l'équation différentielle linéaire, grâce aux grandeurs complexes associées, se réduit à la résolution d'une équation complexe algébrique.

        Si un terme du type $\lambda \int x \diff{}t$ est présent dans l'équation différentielle à résoudre, le plus simple est de dériver les deux membres de l'égalité pour éviter ce type de terme.
\section{Impédance complexe, admittance complexe d'un dipôle linéaire passif}
    On s'intéresse ici aux dipôles linéaires passifs dont la force électromotrice, ou courant électromoteur est nul(le) en régime stationnaire~: conducteurs ohmiques, bobines et condensateurs.

    En régime sinusoïdal permanent, la tension est alternative sinusoïdale, la grandeur complexe associée à cette tension est du type $\uc = \Uc \exp(\ju\omega t)$ et l'intensité (fléchée en sens inverse) est aussi alternative sinusoïdale, dont la grandeur complexe associée est du type $\ic = \Ic \exp(\ju\omega t)$. C'est la solution particulière de l'équation différentielle particulière complexe associée à l'équation différentielle réelle qui relie $u$ et $i$. Cette solution étant obtenue par la méthode exposée dans la section \ref{sec:complexe}, c'est-à-dire par la méthode de résolution algébrique.
    \subsection{Impédance complexe}
        Par définition l'impédance complexe du dipôle considéré est $\Zc = \frac{\Uc}{\Ic}=\frac{\uc}{\ic}$ en $\si{\ohm}$. Son module est l'impédance du dipôle~: $Z = \abs{\Zc} = \frac{U}{I}$. Son argument est le déphasage de $u$ par rapport à $i$~:
        \begin{equation}
            \Phi = \Arg(\Zc) = \Arg(\uc) - \Arg(\ic) = \Arg(\Uc) - \Arg(\Ic).
        \end{equation}
        La partie réelle de l'impédance est la résistance du dipôle $R = \Re(\Zc)$. Sa partie imaginaire est la réactance $S=\Im(\Zc)$. On a donc $\Zc = R + \ju S$, $Z = \sqrt(R^2+S^2)$ et comme $R$ est toujours positif, $\Phi = \arctan(S/R)$, avec $\cos\Phi = R/Z$ et $\sin\Phi=S/Z$.
    \subsection{Admittance complexe}
        Par définition, l'admittance complexe du dipôle considéré est $\Yc = \frac{\Ic}{\Uc}=\frac{\ic}{\uc}=\frac{1}{\Zc}$ en $\si{\siemens}$, "Siemens". Son module est l'admittance du dipôle~: $Y = \abs{\Yc} = \frac{I}{U} = \frac{1}{Z}$. Son argument est le déphasage de $i$ par rapport à $u$~:
        \begin{equation}
            -\Phi = \Arg(\Yc) = \Arg(\ic) - \Arg(\uc) = \Arg(\Ic) - \Arg(\Uc) = -\Arg(\Zc).
        \end{equation}
        Attention, sa partie réelle n'est pas l'inverse de la résistance.
    \subsection{Impédance et admittance des dipôles linéaires simples}
        \begin{itemize}
            \item Résistance pure~: $u=Ri$ donc $\uc = R \ic$, $\Zc =R$, $\Yc = G$, $Z=R$ et $\Phi=0$, $u$ et $i$ sont en concordance de phase ;
            \item Auto-inductance pure~: $u=L\derived{i}{t}$, donc $\uc = \ju L \omega \ic$, donc $\Zc = \ju L\omega$ et $\Yc = \frac{-\ju}{L\omega},$ $Z=L\omega$ et $\Phi=\frac{\pi}{2}$. $u$ est en quadrature avance sur $i$.
            \item Capacité pure~: $i=C\derived{u}{t}$, d'où $\ic = \ju\omega C \ic$, donc $\Yc = \ju\omega C$ et $\Zc = \frac{-\ju}{C\omega}$, $Y=C\omega$, $Z=\frac{1}{C\omega}$ et $\Phi = -\frac{\pi}{2}$. $u$ est en quadrature retard sur $i$.
        \end{itemize}
    \subsection{Comportement fréquentiel des dipôles linéaires simples}
        Pour une auto-inductance pure, lorsque $\omega$ tend vers 0, l'impédance $Z$ tend vers 0 et donc la tension $U$ tend vers 0, pour toute intensité $I$. \emph{Une auto-inductance pure à basse fréquence, ou en régime stationnaire, est un court-circuit}.

        Pour une auto-inductance pure, lorsque $\omega$ tend vers l'infini, alors l'admittance $Y$ tend vers 0, donc $I$ tend vers 0 pour toute tension $U$. \emph{Une auto-inductance pure à haute fréquence est un coupe-circuit}.

        Pour une capacité pure, lorsque $\omega$ tend vers 0 alors l'admittance $Y$ tend vers 0 et $I$ tend vers 0, pour toute tension $U$. \emph{Une capacité pure à basse fréquence, ou en régime stationnaire, est un coupe-circuit}.

        Pour une capacité pure, lorsque $\omega$ tend vers l'infini, alors l'impédance $Z$ tend vers 0, donc $U$ tend vers 0 pour toute intensité $I$. \emph{Une capacité pure à haute fréquence est un court-circuit}.
\section{Lois de Kirchhoff, associations de dipôles, équivalences}
    \subsection{Lois de Kirchhoff, théorème de Millman}
        En régime sinusoïdal permanent (quasi-stationnaire donc de fréquence pas trop élevée), la loi des n\oe{}ds et la loi des mailles s'appliquent aux valeurs instantanées des intensités et des tensions~:
        \begin{equation}
            \label{eq:noeuds_cos}
            \sum_{k} I_k\cos(\omega t +\varphi_k) = 0.
        \end{equation}
        En reculant l'origine des dates de $T/4$, on en déduit~:
        \begin{equation}
            \label{eq:noeuds_sin}
            \sum_{k} I_k\sin(\omega t +\varphi_k) = 0.
        \end{equation}
        En additionnant ces deux équations, $\eqref{eq:noeuds_cos} + \ju \eqref{eq:noeuds_sin}$, il vient~:
        \begin{equation}
            \label{eq:noeuds_cplx}
            \sum_k \Ic_k \exp(\ju\omega t) = 0
        \end{equation}
        La loi des noeuds s'applique donc aux grandeurs complexes associées aux intensités et aux amplitudes complexes des intensités~: Pour l'ensemble des $n$ courants arrivant en un noeud $\sum_{k=1}^n \ic_k = 0$ et $\sum_{k=1}^n \Ic_k = 0$.

        On démontre de même que la loi des mailles en régime sinusoïdal permanent~: La loi des mailles s'applique donc aux grandeurs complexes associées aux tensions et aux amplitudes complexes des tensions~: Pour l'ensemble des $n$ tensions représentées par des flèches, correspondant toutes aux même sens de parcours le long d'une maille $\sum_{k=1}^n \uc_k = 0$ et $\sum_{k=1}^n \Uc_k = 0$.

        Pour un dipôle linéaire passif, la définition de l'impédance complexe et celle de l'admittance complexe donne l'équivalent en régime sinusoïdal permanent de la loi d'Ohm~:
        \begin{align}
            \label{eq:ohm_cplxZ}    \uc = \Zc ~ \ic &\quad \Uc = \Zc ~ \Ic \\
            \label{eq:ohm_cplxY}    \ic = \Yc ~ \uc &\quad \Ic = \Yc ~ \Uc.
        \end{align}
        En combinant \eqref{eq:ohm_cplxY} avec la loi des noeuds, on obtient le théorème de Millman~: en notant $\Yc = \sum_{k=1}^n \Yc_k$ alors
        \begin{align}
            \uc & = \sum_{k=1}^n \uc_k \frac{\Yc_k}{\Yc} + \sum_{j=1}^m \frac{\ncomplexe{\eta}_j}{\Yc} \\
            \Uc & = \sum_{k=1}^n \Uc_k \frac{\Yc_k}{\Yc} + \sum_{j=1}^m \frac{\ncomplexe{H}_j}{\Yc}.
        \end{align}
    \subsection{Associations de dipôles linéaires}
	    \paragraph{En parallèle} Les dipôles sont sous la même tension et les intensités s'ajoutent. Pour les grandeurs complexes associées, on a donc~: $\ic = \sum_k {\ic}_k = \sum{k} {\yc}_k \uc $. \emph{L'admittance complexe du dipôle équivalent est la somme des admittances complexe des dipôles en parallèle}~: $\Yc = \sum_{k=1}^n \Yc_k$. En particulier pour deux dipôles en parallèle : $\Yc = \Yc_1 + \Yc_2$, d'où $\Zc = \frac{\Zc_1 \Zc_2}{\Zc_1+\Zc_2}$. Par exemple, l'admittance complexe d'un condensateur de capacité $C$ avec une conductance de fuite non nulle $G$ est~: $\Yc = G + \ju\omega C$, donc $Y = \sqrt{G^2 + C^2 \omega^2}$, d'où $Z = \frac{R}{\sqrt{1+(RC\omega)^2}}$ et le déphasage de la tension sur l'intensité est $\Phi = -\arctan\left(\frac{C\omega}{G}\right) = -\arctan(RC\omega)$.
	    
	    \paragraph{En série} Les dipôles sont traversés par le même courant et les tensions s'ajoutent. Pour les grandeurs complexes associées, on a donc~: $\uc = \sum_k \Zc_k \ic$. \emph{L'impédance complexe du dipôle équivalent est la somme des impédances complexes des dipôles en série~: $\Zc = \sum_k \Zc_k$. En particulier pour deux dipôles en série : $\Zc = \Zc_1 + \Zc_2$, d'où $\Yc = \frac{\Yc_1 \Yc_2}{\Yc_1+\Yc_2}$. Par exemple, l'impédance complexe d'une bobine réelle, si l'on peut négliger sa capacité est~: $\Zc = R + \ju\omega L$, donc $Z = \sqrt{R^2+L^2\omega^2}$, d'où $Y = \frac{G}{\sqrt{1+(GL\omega)^2}}$, avec $G=1/R$, et le déphasage de la tension sur l'intensité est $\Phi = \arctan\left(\frac{L\omega}{R}\right) = \arcten(GL\omega)$.
    	\paragraph{Théorème de Kennely}
    		Il y a équivalence entre le triangle et l'étoile et on démontre, comme en régime continu~:
    	\begin{align}
    		\Zc_{12} = \frac{\Zc_1\Zc_2}{\Zc_1 + \Zc_2 + \Zc_3}  & \Zc_{23} = \frac{\Zc_2\Zc_3}{\Zc_1 + \Zc_2 + \Zc_3} & \Zc_{31} = \frac{\Zc_3\Zc_1}{\Zc_1 + \Zc_2 + \Zc_3} \\
    		\Yc_1 = \frac{\Yc_{12}\Yc_{31}}{\Yc_{12} + \Yc_{23} + \Yc_{31}}  & \Yc_2 = \frac{\Yc_{23}\Yc_{12}}{\Yc_{12} + \Yc_{23} + \Yc_{31}} & \Yc_3 = \frac{\Yc_{31}\Yc_{23}}{\Yc_{12} + \Yc_{23} + \Yc_{31}}
    	\end{align}
    	%
    \subsection{Modèle de Norton et modèle de Thévenin d'un dipôle linéaire  actif}	
    	Les équations de ces modèles sont~: $e = E\cos(\omega t + \varphi)$ et $\eta = H\cos(\omega t + \psi)$ étant respectivement la force électromotrice et le courant électromoteur du dipôle linéaire, en régime sinusoïdal permanent de pulsation $\omega$. $\Zc$ et $\Yc$ respectivement l'impédance complexe et l'admittance complexe de ce dipôle, on a~: $u = u_Z -e$, d'où $\uc = \Zc \ic -\ncomplexe{e}$ et en divisant par $\exp(\ju\omega t)$, on obtient $\Uc = \Zc \Ic - \ncomplexe{E}$. On a aussi~: $i = i_Y + \eta$, d'où $\ic = \ic_Y + \ncomplexe{\eta}$ et en divisant par $\exp(\ju\omega t)$, on obtient $\Ic = \Yc \Uc + \ncomplexe{H}$.
    
    	\emph{Les deux modèle représentent le même dipôle si et seulement si~: $\Yc = 1/\Zc$ et $\ncomplexe{eta} = \ncomplexe{e}/ \ncomplexe{Z}$, soit $\ncomplexe{H} = \Ec/\Zc$.} On a donc pour les amplitudes, $H = E/Z = YE$ et pour les phases initialement à $t=0$, $\psi = \varphi - \Arg(\Zc) = \varphi+\Arg(\Yc)$.
    	%
    \subsection{Pont diviseur de tension}
    	En sortie ouverte~: $\frac{\uc_s}{\uc_e} = \frac{\Uc_s}{\Uc_e} = \frac{\Zc_2}{\Zc_1+\Zc_2}$, et aussi en divisant par $\Zc_1 \Zc_2$ : $frac{\Uc_s}{\Uc_e} = \frac{\Yc_1}{\Yc_1+\Yc_2}$.
    	
    	Avec une impédance de charge, il faut remplacer $\Zc_2$ par l'impédance équivalente $\Zc_2 \parallel \Zc_c$, ce qui donne $\frac{\Uc_s}{\Uc_e} = \frac{\Zc_c \Zc_2}{\Zc_1\Zc_2 + \Zc_c\Zc_1 + \Zc_c\Zc_2} = \frac{\Yc_1}{\Yc_1+\Yc_2+\Yc_c}$
    	%
	\subsection{Pont diviseur de courant}
		En sortie court-circuitée~: $\frac{\ic_s}{\ic_e} = \frac{\Ic_s}{\Ic_e} = \frac{\Yc_2}{\Yc_1+\Yc_2}$, et aussi en divisant par $\Yc_1 \Yc_2$ : $frac{\Ic_s}{\Ic_e} = \frac{\Zc_1}{\Zc_1+\Zc_2}$.
		
		Avec une conductance de charge, il faut remplacer $\Yc_2$ par la conductance équivalente $\Yc_2 \parallel \Yc_c$, ce qui donne $\frac{\Ic_s}{\Ic_e} = \frac{\Yc_c \Yc_2}{\Yc_1\Yc_2 + \Yc_c\Yc_1 + \Yc_c\Yc_2} = \frac{\Zc_1}{\Zc_1+\Zc_2+\Zc_c}$
\section{Exercices}
	\begin{exercice}[Circuit complexe en régime sinusoïdal permanent]
		Le réseau représenté ci-dessous est alimenté par une source de tension alternative $E = E \cos(\omega t). On donne $R=$\SI{10}{\ohm}$ et $E=\SI{20}{\volt}$. La fréquence du générateur est réglé de façon que $L\omega = \frac{1}{C\omega}=R$. Calculer l'amplitude $I$ de l'intensité du courant dans la résistance.
	\end{exercice}
	\begin{exercice}[Dipôle $R, L, C$ parallèle]
		Dans le montage ci-dessous, le générateur délivre une tension sinusoïdale de pulsation $\SI{1000}{\rad\per\second}$ d'amplitude $\SI{10}{\volt}$. On constate que l'intensité du courant débité par le générateur prend la même valeur efficace pour deux valeurs différentes de capacité~: $\SI{3,2}{\micro\farad}$ et $\SI{6,9}{\micro\farad}$.
		\begin{itemize}
			\item Déterminez $L$. \item Pour quelle valeur de $C$ (notée $C_0$) l'intensité efficace sera minimale ?
		\end{itemize}
	\end{exercice}