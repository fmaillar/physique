\chapter{Conducteurs ohmiques, loi d'ohm, loi de joule}
\minitoc
\minilof
\minilot

\section{Conductivité, loi d'Ohm locale}
\label{chap10-sec:conductivite}

\subsection{Conducteur ohmique, mobilité des porteurs de charge}
\label{chap10-subsec:conducteurohmique}

On considère un conducteur homogène, électriquement isotrope, immobile et à température constante et uniforme. 
Dans un tel conducteur, le mouvement d'ensemble des porteurs de charge est dû à une seule cause, le champ électrique créé par le reste du circuit.
Le conducteur étant isotrope, il n'existe aucune direction privilégiée, mise à part celle du champ électrique. La vitesse moyenne (vitesse de leur mouvement d'ensemble) de chaque type de porteurs de charge est donc parallèle à $\vect{E}$. Les lignes du champ électrique sont donc les lignes de courant.
La \emph{mobilité} $\mu_i$ du $i$\ieme type de porteurs de charge est définie par $\vv_i= \mu_i \vect{E}$ lorsque le régime stationnaire (ou quasi-stationnaire) est atteint. La mobilité $\mu_i > 0$ si $q_i > 0$ et $\mu_i < 0$ si $q_i < 0$.

La limitation de la vitesse moyenne d'un porteur de charge est due aux interactions avec les obstacles. Dans un conducteur métallique, ces obstacles sont les défauts du réseau cristallin et les impuretés, c'est-à-dire tout ce qui rompt la périodicité spatiale du cristal. Dans un électrolyte ce sont les ions de charge opposée et éventuellement les molécules du solvant.
La mobilité d'un porteur de charge dépend de la température. Pour un métal elle décroît si la température $T$ croît, car les défauts cristallins sont en nombre croissant avec $T$. Par contre, pour un électrolyte, elle croît avec $T$.
La densité volumique de courant est $\vj = \sum_i \rho_i \vv_i = \sum_i \rho_i \mu_i \vect{E}$.

\subsection{Loi d'Ohm locale}
\label{sec:loidOhmlocale}

Un conducteur suit la  loi d'Ohm locale si et seulement si $\mu_i$  est indépendant de $\vect{E}$.

Les métaux suivent la loi d'Ohm locale, même pour des champs électriques intenses. Dans ce cas, le conducteur considéré est un conducteur ohmique. Sa conductivité est définie par $\rho = \sum_i \rho_i \mu_i$ . Elle est indépendante de $\vect{E}$ et uniforme dans tout le conducteur. Elle est de plus indépendante du temps en régime stationnaire ou elle en dépend très peu en ARQS. Son unité internationale est le siemens par mètre : $\si{S/m}$.

La loi d'Ohm locale se traduit donc par la formule $\vj = \sigma \vect{E}$ avec $\sigma$ constant. L'inverse de la conductivité est la résistivité , son unité internationale est l'ohm-mètre : $\si{\ohm.m}$. Le siemens est l'inverse de l'ohm. La loi d'Ohm locale s'écrit donc encore $\vect{E} = \rho \vj$.

\section{Résistance électrique d'un conducteur ohmique, loi d'Ohm}
\label{chap10-sec:resistanceelectrique}

\subsection{Loi d'Ohm}
\label{chap10-subsec:loidohm}

Pour un conducteur ohmique, c'est-à-dire un conducteur homogène, isotrope, immobile, en régime stationnaire ou en ARQS (donc en particulier à $T$ constante), la loi d'Ohm s'écrit : $u = R i$ avec $R$ constante, si l'on utilise la convention récepteur. 
%mettre une figure

$R$ est la résistance du conducteur ohmique, son unité SI est l'ohm : $\si{\ohm}$. Son inverse est la conductance $G=\frac{1}{R}$. Son unité SI est le siemens $\si{S}$. La loi d'Ohm s'écrit donc aussi $i = G u$. La loi d'Ohm est bien entendue une conséquence de la loi d'Ohm locale~: Soit un conducteur ohmique traversé par un courant en régime stationnaire, limité par deux équipotentielles A et B. Sa surface latérale est un tube de courant puisque aucun courant n'en sort. S étant une section quelconque du conducteur, l'intensité du courant est $ i = \iint_{S} \vj \cdot \vect{\D S} = \sigma \iint_S \vj \cdot \vec{\D S}$ et la tension est $u \int_A^B \vect{E} \cdot \vect{\D M}$. Si en chaque point du conducteur, $\vect{E}$ est multiplié par $k$, alors $u$ est multiplié par $k$ et $\sigma$ étant inchangé, $i$ est aussi multiplié par $k$. Le rapport $R = \frac{u}{i}$ reste bien constant.

\subsection{Résistance d'un conducteur ohmique élémentaire et d'un conducteur ohmique cylindrique}

Soit un tube de courant élémentaire de longueur $\D L$ et de section $\D S$, limité par deux équipotentielles entre lesquelles la tension est $\D u$ et parcouru par le courant d'intensité $\D i$. 

Le champ électrique $\vect{E}$ est comme les lignes de courant normal aux équipotentielles et il est donc parallèle au vecteur $\D L$ et au vecteur $\D S$. 

On a alors $\D i = \sigma E \D S$ et $\D u = E \D L$ d'où la conductance de cet élément de conducteur ohmique~: $\D G = \sigma \derived{S}{L}$.
 (c'est un infiniment petit) et sa résistance (infiniment grande)~: $\Delta R = \rho \derived{L}{S}$.

Par intégration, on obtient la résistance d'un conducteur ohmique cylindrique limité par deux équipotentielles, si la densité volumique de courant est bien parallèle à l'axe du cylindre : $R = \rho \frac{L}{S}$.

Si un conducteur est filiforme, de section constante, chaque petite portion du fil est assimilable à un cylindre, la formule ci-dessus s'applique aussi.

Pour une forme quelconque du conducteur ohmique, on peut calculer sa résistance ou sa conductance en le décomposant en tubes de courant élémentaires et en appliquant les lois sur les associations de résistances ou de conductances.

\section{Étude physique de la conductivité}

\subsection{Cas des métaux, des alliages métalliques}

Pour un métal pur~:
\begin{itemize}
\item aux températures ordinaires : $\rho = \rho_0 (1 + a \theta)$, avec $\theta$ représentant la température en $\degrees$C et $a$ est de l'ordre de $\SI{3,7e-3}{K^{-1}}$;
\item à des températures basses ($\SI{20}{K}$ à $\SI{100}{K}$ environ) la résistivité varie plus rapidement et non linéairement.
\end{itemize}

Les alliages métalliques ont souvent des coefficients a bien plus faibles, parfois même très faibles (constantan, manganine).

\subsection{Supraconducteurs}

Il existe une température critique en dessous de laquelle la résistivité s'annule pour certains matériaux. En dessous de cette température critique, le matériau est ``supraconducteur''.

Pour les métaux, la température critique est toujours très basse~: quelques kelvins pour la plupart, mais il n'y a pas de température critique pour les meilleurs conducteurs (cuivre, argent).

\subsection{Électrolytes}

La conductivité d'un électrolyte est une fonction croissante de la température, (la mobilité des ions croît avec la température).

Soit un électrolyte dans lequel les porteurs de charge sont les ions libres $X_i^{z_i^+}$($z_i > 0$ ou $z_i < 0$ suivant si l'on a affaire à un cation ou à un anion).

La charge de l'ion est $q_i = z_i e$. La concentration molaire volumique de cet ion étant $c_i$ (ou $[X_i^{z_i^+}]$ ), la concentration volumique de ces ions est $n_i = N  c_i$. La densité volumique de charge mobile pour ces ions est donc $\rho_i = N  c_i z_i e$. En utilisant la constante de Faraday $F = N e = \SI{96,5}{kC.mol^{-1}}$, on obtient : $\rho_i = z_i F c_i$. On a donc~: .
\begin{equation}
  \vj = \sum_{i} \rho_i \vv_i = \sum_i z_i \mu_i c_i \vect{E}
\end{equation}

Avec la loi d'Ohm locale $\vj = \sigma \vect{E}$, on obtient  $\sigma = \sum_i z_i \mu_i c_i$

La conductivité molaire des ions $X_i^{z_i^+}$  est définie par $\lambda_i = z_i \mu_i F$, son unité SI est : $\si{S.m^2.mol^{-1}}$, $\lambda_i > 0$ car $z_i$ et $\rho_i$ sont de même signe. D'où l'expression de la conductivité de l'électrolyte : $\sigma = \sum_i \lambda_i c_i$. Les conductivités molaires des ions, comme leurs mobilités, dépendent de la température et des concentrations.

Pour une solution suffisamment diluée, la conductivité molaire de chaque ion tend vers une limite appelée conductivité molaire limite, notée $\lambda^0$, qui croît avec la température. Pour une solution très diluée : 
\begin{equation}
  \sigma = \sum_i \lambda_i^0 c_i
\end{equation}

\section{Associations de résistances}

\subsection{Association en série}

Des dipôles sont en série s'ils sont traversés par le même courant. Soit $i$ l'intensité de ce courant, $R$ la résistance électrique du $k$\ieme{} conducteur et $u_k = R_k i$ la tension entre les équipotentielles qui le limitent. La tension entre les bornes de l'association des $n$ conducteurs ohmiques est $u = \sum_{k=1}^n u_k = \sum_{k=1}^n R_k i$. La résistance du conducteur ohmique équivalent au groupement en série est donc $R = \sum_{k=1}^n R_k$.

\subsection{Association en parallèle}

Des dipôles sont en parallèle s'ils sont placés entre les deux mêmes équipotentielles, donc sous la même tension. Soit $u$ la tension entre ces deux équipotentielles, $G_k$ la conductance électrique du $k$\ieme{} conducteur et $i_k = G_k u$ l'intensité du courant qui le traverse. L'intensité du courant qui traverse l'ensemble des $n$ conducteurs ohmiques est $i = \sum_{k=1}^n i_k = \sum_{k=1}^n G_k u$. La conductance du conducteur ohmique équivalent au groupement en parallèle est donc $G = \sum_{k=1}^n G_k$.

\subsection{Cas de deux conducteurs ohmiques}

Pour deux conducteurs ohmiques en série $R=R_1+R_2$ et $G = \frac{G_1G_2}{G_1+G_2}$.

Pour deux conducteurs ohmiques en parallèle $R = \frac{R_1R_2}{R_1+R_2}$ et $G=G_1+G_2$.

\subsection{Théorème de Kennely (équivalence triangle, étoile)}

On admettra qu'il y a équivalence entre les deux réseaux ci-dessous et on cherchera les relations entre les résistances de l'étoile et celles du triangle.

