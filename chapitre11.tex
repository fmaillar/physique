\chapter{Dipôles électrocinétiques}
\minitoc
\minilof
\minilot

\section{Dipôle électrocinétique, puissance électrocinétique}
\subsection{Circuits étudiés}

On ne considérera ici que des circuits dans lesquels les dipôles seront reliés entre eux par des fils que l'on assimilera à de simples lignes conductrices, le plus souvent, de résistances négligeables (de conductances infinies). La tension entre les deux extrémités d'un fil sera alors nulle. Un dipôle sera donc limité par deux équipotentielles assimilables à des points. Chaque dipôle sera donc couplé électriquement au reste du circuit. Avec la convention récepteur, la puissance électrocinétique que reçoit un dipôle de la part du reste du circuit est $P = ui$.

La charge entrant par A pendant dt est $\D q = i \D t$, la charge sortant par B pendant $\D t$ est la même (ARQS). L'énergie électrocinétique reçue par le dipôle pendant dt est  $\delta W = P \D t = u i \D t = u \D q$. Avec la convention générateur (flèche u inversée), on a bien sûr $P  = - u i$  et  $\delta W = - u i \D t$. On se placera toujours dans des conditions telles que l'on puisse appliquer l'approximation des régimes quasi stationnaires. La température d'un dipôle sera considérée comme constante, sa résistance le sera donc aussi. Toute l'énergie thermique produite par effet Joule sera donc évacuée sous forme de chaleur $P_{th} = R i^2$ et  $\delta W_{th} = R i^2 \D t = -\delta Q$, si $R$ est la résistance du dipôle.

\subsection{Transformations d'énergie dans un dipôle}
En plus du couplage électrique avec le reste du circuit et du couplage thermique avec l'extérieur, le dipôle peut être couplé au milieu extérieur de différentes façons~:
\begin{itemize}
\item Couplage mécanique pour un moteur ou un alternateur comme une transformation d'énergie électrocinétique en énergie mécanique ou l'inverse ;
\item Couplage chimique pour un électrolyseur ou une pile comme une transformation d'énergie électrocinétique en énergie chimique ou l'inverse ;
\item Couplage par rayonnement  dans le cas d'un photopile ;
\item Couplage électromagnétique \ldots
\end{itemize}

Dans tous les cas l'énergie électrocinétique reçue sera intégralement transformée en autres formes d'énergie :
\begin{itemize}
\item en énergie thermique uniquement pour un conducteur ohmique ;
\item en énergie thermique et en une autre forme d'énergie (chimique, mécanique, électromagnétique), cédée à l'extérieur ou accumulée dans le dipôle pour les autres dipôles.
\end{itemize}

Ceci est valable au sens algébrique du terme, un récepteur reçoit de l'énergie électrocinétique alors qu'un générateur peut en fournir au reste du circuit. On reviendra sur ces notions de générateur et de récepteur plus loin dans ce chapitre.

\section{Caractéristique courant tension d'un dipôle}
\subsection{Définition}

Lorsque la caractéristique existe, c'est la relation (éventuellement la fonction) qui à $i$ fait correspondre $u$, ou celle qui à $u$ fait correspondre $i$. Un dipôle possédant une caractéristique est linéaire si sa caractéristique est une droite. Il est symétrique si $u(-i) = - u(i)$ (fonction impaire). Les deux bornes du dipôle sont alors équivalentes.
	
Il est actif si pour $i = 0$, $u \neq 0$. Lorsqu'il est passif, sa caractéristique passe par l'origine des axes. Le point de fonctionnement d'un dipôle est le point de coordonnées $(u,i)$ correspondant à son fonctionnement dans le circuit considéré. Beaucoup de dipôles n'ont pas de caractéristique, c'est par exemple le cas si la relation entre $u$ et $i$ est une équation différentielle comme par exemple pour une bobine $u = L \derived{i}{t}$.

\subsection{Exemples de caractéristiques de dipôles}

\subsection{Dipôles linéaires}

Un dipôle est linéaire si et seulement si $u$ est une fonction affine de $i$ ou est lié à $i$ par une équation différentielle linéaire. En régime stationnaire, les dipôles linéaires sont les conducteurs ohmiques ($u = R i$) et les générateurs et récepteurs linéaires ($u = R i - e$ avec la convention récepteur). En régime quasi-stationnaire il faut ajouter les bobines et les condensateurs.

\section{Dipôles linéaires idéaux}
\subsection{Sources de tension}

Ce sont des générateurs (ou récepteurs) idéaux pour lesquels la tension entre les bornes est indépendante de l'intensité du courant qui les traverse. Le tension $e$ est la force électromotrice (force électro-motrice) de la source de tension. C'est la tension constante entre ses bornes, fléchée habituellement dans le même sens que le courant $i$. C'est un dipôle actif, non symétrique, ses deux bornes (ou pôles) sont différents : Si e > 0, le pôle + est du côté de la pointe de la flèche représentant la force électro-motrice Avec la convention récepteur, l'équation de la caractéristique est quelque soit le courant $i$ : $u=-e$. La puissance électrocinétique est $P=-ei$. 

Son opposé est la puissance électrocinétique engendrée : $P' = e i$. Lorsqu'elle est positive, ($e$ et $i$ de même signe), la source transforme de l'énergie chimique, mécanique ou autre en énergie électrocinétique qu'elle fournit au reste du circuit. Il s'agit donc alors d'un générateur. Si c'est $P$ qui est positive ($P ' < 0$), elle transforme de l'énergie électrocinétique qu'elle reçoit du reste du circuit en énergie chimique, mécanique ou autre. Il s'agit alors d'un récepteur.

\subsection{Sources de courant}

Ce sont des générateurs (ou récepteurs) idéaux pour lesquels l'intensité du courant qui les traverse est indépendante de la tension entre les bornes.

$\eta$ est le courant électromoteur (c.é.m.) de la source de courant. C'est l'intensité constante qui traverse la source de courant, fléchée habituellement dans le même sens que $i$.
C'est un dipôle actif, non symétrique, ses deux bornes (ou pôles) sont différents : Si $\eta > 0$, le pôle + est du côté de la pointe de la flèche représentant le c.é.m. Avec la convention récepteur, l'équation de la caractéristique est quelque soit la tension $u$ : $i=\eta$. La puissance électrocinétique reçue est $P  = u i = u \eta$. Lorsqu'elle est négative, ($u$ et $\eta$ de signe opposés), la source transforme de l'énergie chimique, mécanique ou autre en énergie électrocinétique qu'elle fournit au reste du circuit. Il s'agit donc alors d'un générateur. Son opposé est la puissance électrocinétique engendrée : $P' = -u\eta$. Si c'est $P'$ qui est positive ($P  < 0$), elle transforme de l'énergie électrocinétique qu'elle reçoit du reste du circuit en énergie chimique, mécanique ou autre. Il s'agit alors d'un récepteur.

\subsection{Auto-inductance pure}

Il s'agit du cas idéal d'une bobine de résistance nulle. Dans le cours de deuxième année, on démontrera que si le courant $i$ varie, la bobine est le siège d'un phénomène d'auto-induction. La force électro-motrice d'auto-induction est $e = -L\derived{i}{t}$ . La relation entre $u$ et $i$ est donc l'équation différentielle linéaire : $u = L\derived{i}{t}$. C'est donc une source de tension variable puisque sa force électro-motrice dépend des variations de i au cours du temps.

\subsection{Capacité pure}

Il s'agit du cas idéal d'un condensateur parfait, c'est-à-dire parfaitement isolant (de conductance nulle). Par définition de la capacité d'un condensateur, $q=Cu$. $q$ étant la charge accumulée sur l'armature où arrive le courant d'intensité $i$ donc $i = \derived{q}{t}$  et la relation entre $i$ et $u$ est l'équation différentielle linéaire : $i=C\derived{u}{t}$. On peut considérer que c'est une source de courant variable puisque son c.é.m. dépend des variations de u au cours du temps.

\section{Bobines et condensateurs réels}

\subsection{Bobine de résistance non négligeable}

On peut l'assimiler à une auto-inductance pure en série avec un conducteur ohmique. En réalité, il faudrait modéliser la bobine en ajoutant en parallèle une capacité due au vernis isolant qui se trouve entre les spires. Mais cette capacité est négligeable pour des fréquences pas trop élevées. D'autre part, pour une bobine avec noyau de fer, si le courant est intense, $L$ varie avec $i$ et le dipôle n'est donc plus linéaire. Par addition des tensions aux bornes des deux dipôles élémentaires imaginaires $R$ et $L$, on a donc : $u = L\derived{i}{t} + Ri$.

\subsection{Condensateur réel avec conductance de fuite}

Si le diélectrique d'un condensateur n'est pas un isolant parfait, un courant le traverse. On peut alors le modéliser comme une capacité pure en parallèle avec un conducteur ohmique.
La loi des nœuds donne : $i = Gu + C \derived{u}{t}$.

\section{Générateurs et récepteurs linéaires}
Tout générateur ou récepteur linéaire peut être modélisé de deux façons différentes.
\subsection{Modèle de Thévenin}
Avec la convention récepteur et $e$ fléché dans le même sens que $i$, l'équation de la caractéristique est $u=Ri-e$. Si le pôle + est du côté de la pointe de la flèche $e$, alors $e > 0$. C'est à ce cas que correspond le tracé ci-dessus.
\subsection{Modèle de Norton}
Avec la convention récepteur et $\eta$ fléché dans le même sens que $i$, l'équation de la caractéristique est $i=Gu+\eta$. Si le c.é.m. $\eta$ est fléché comme sortant du pôle +, alors $\eta > 0$. C'est à ce cas que correspond le tracé ci-dessus.
\subsection{Équivalence des deux modèles}
Les deux modèles représentent donc le même dipôle si et seulement si
$G=\frac{1}{R}$ et $e = R \eta$. La tension à vide est la valeur de $u$ pour $i =
0$, (coupe circuit dans sa branche).  On la mesure en plaçant directement un
voltmètre (de résistance infinie) entre les bornes du dipôle non connecté à un
circuit. Sa valeur est $-e = -R\eta = -\frac{\eta}{G}$.

L'intensité du courant de court-circuit est la valeur de $i$ pour $u = 0$
(court-circuit réalisé en reliant les deux bornes avec un conducteur ohmiques de
résistance négligeable). On la mesure en plaçant directement un ampèremètre (de
conductance infinie) entre les bornes du dipôle. Sa valeur est $\eta = Ge =
\frac{e}{R}$.
\subsection{Puissance électrocinétique reçue, puissance électrocinétique
engendrée, différents types de fonctionnement}

La puissance électrocinétique reçue par le dipôle est  $P  = u i = R i^2 - e i = P_{th} - P'$. Il cède donc au reste du circuit  $-P  = P' -P{th}$ : il cède la puissance électrocinétique qu'il engendre diminuée de celle qu'il consomme par effet Joule. Toutes les grandeurs sont en fait algébriques dans ces relations, sauf  bien sûr $P_{th} > 0$.
	
Étudions le cas où $e > 0 (\eta > 0)$ ; il y a trois fonctionnements possibles :
Si $i > 0$  et $u < 0$, alors $P'> 0$ et $-P  > 0$ , avec $-P  < P'$. C'est le cas habituel d'un générateur qui engendre effectivement de la puissance électrocinétique, qui en fournit une partie au reste du circuit et qui dissipe le reste par effet Joule.

Si $i < 0$, et $u < 0$, alors $P > 0$ et $P' < 0$ . C'est le cas d'un fonctionnement en récepteur; le dipôle consomme de la puissance électrocinétique par effet joule et par un autre effet (chimique, mécanique...). Ce récepteur peut être un "générateur monté en opposition", $i < 0$ étant imposé par un autre générateur de force électro-motrice plus grande (en valeur absolue), les deux pôles + étant reliés entre eux.

Si $i > 0$ et $u > 0$, alors $P > 0$ et $P' > 0$ avec  $P > P'$. On a dans ce cas $i > \eta$, l'intensité du courant est plus grande que celle du courant de court-circuit (avec souvent un risque de détérioration du fait d'un effet Joule intense). C'est le cas exceptionnel d'un générateur qui engendre effectivement de la puissance électrocinétique mais qui en consomme par effet Joule plus qu'il n'en engendre.  Ce n'est possible que s'il est monté en série avec un autre générateur de force électro-motrice grande et de résistance petite (son pôle – est relié au pôle + de l'autre).
\subsection{Cas d'un dipôle non linéaire possédant une caractéristique}
C'est en fait le cas habituel pour les générateurs et récepteurs réels. On peut, au voisinage du point de fonctionnement assimiler la caractéristique à sa tangente et écrire suivant la modélisation choisie $u = R i - e$ ou i$ = G u + \eta$. La caractéristique est "linéarisée". Mais alors, $R$ (ou $G$) et $e$ (ou $\eta$) dépendent du point de fonctionnement.
\subsection{Récepteurs (électrolyseurs, moteurs électriques) non polarisés}
Il s'agit de récepteurs symétriques, leurs deux bornes sont identiques. 

$E' > 0$ est la force contre électromotrice (f.c.é.m.) du récepteur. La caractéristique comporte trois parties linéaires~:
\begin{align}
i > 0 & u = R i + E' \ (u > E') \\
i < 0 & u  = R i - E' \ (u < -E') \\
i = 0 &  - E' < u < E'.
\end{align}

La puissance utile, c'est-à-dire la puissance électrocinétique transformée en puissance chimique ou mécanique est $P_u = E' \abs{i}$. Elle est toujours positive. La puissance électrocinétique reçue est $P = u i = R i^2 + E' \abs{i} = P_{th} + P_u$.

\section{Loi d'Ohm généralisée, signe de la force électro-motrice}
Finalement, tout récepteur ou générateur polarisé ou non, si sa caractéristique est linéaire dans le domaine du point de fonctionnement, peut être modélisé par la représentation de Thévenin ou par celle de Norton et suit la loi d'Ohm généralisée qui s'écrit, avec la convention récepteur avec $i$, $e$, et $\eta$ fléchés dans le même sens : $u = R i - e$ ou $i = G u + \eta$.

Si le dipôle est polarisé, alors le signe de $e$ est constant : avec $e$ fléché de $A$ vers $B$, $e = eAB$ on a $e > 0$ si $B$ est le pôle plus, et $e < 0$ si B est le pôle moins. De même pour le signe de $\eta$.

Si le dipôle n'est pas polarisé, alors le signe de $e$ (ou de $\eta$) change avec celui de $i$ : si $i > 0$, $e = –E'$. Si $i < 0$, $e = E'$ . On a bien dans ces deux cas $u = R i - e$.

La puissance utile s'écrit donc aussi $P_u = - e i$ . Elle est toujours positive car un récepteur non polarisé ne peut fonctionner qu'en récepteur. On a donc toujours $ei < 0$.

\section{Générateurs et récepteurs}
Avec la loir d'Ohm généralisée (convention récepteur, $e$ fléchée dans le même sens que $i$)~: $u=Ri-e$. La puissance électrocinétique reçue du reste du circuit est 
\begin{equation}
	\P = ui = Ri^2 - ei = \P_{th}+ \P_u = \P_{th} - \P', 
\end{equation}
où $\P_{th}$ représente la puissance dissipée par effet Joule, $\P_u=-ei$ la puissance dissipée par un autre effet (mécanique, chimique, \ldots{}) et $\P'$ la puissance électrocinétique engendrée dans le dipôle.

Si $\P_u$ est positive, alors le dipôle est un récepteur, si elle est négative, alors le dipôle est un générateur et si elle est nulle il s'agit d'un conducteur ohmique.

Pour un dipôle polarisé, le signe de $e$ est fixé, celui de $i$ dépend de la présence ou non d'autres dipôles polarisés dans le circuit, de leurs sens de branchement (série ou opposition), de leur forces électromotrices et de leurs résistances. Il peut fonctionner compme générateur ou comme récepteur.

Pour un dipôle non polarisé, si $e$ n'est pas nulle, elle est toujours du signe opposé à $i$, il ne peut fonctionner que comme récepteur et ceci uniquement s'il est placé sous une tension suffisante (de valeur absolue supérieure à sa force contre-électromotrice).

Un dipôle non polarisé peut être polarisé lors de son fonctionnement en récepteur, il suffit qu'il accumule au moins une partie de l'énergie non-électrique (chimique, mécanique, électro-magnétique, \ldots) qu'il produit. Il peut alors retransformer cette énergie en énergie électrocinétique, c'est-à-dire, fonctionner aussi en générateur.

\section{Exercices}
\begin{exercice}[Associations de dipôles]
	\begin{enumerate}
		\item Donner l'allure de la caractéristique $i=f(u)$ de deux diodes Zéner en série tête bêche.
		\item Tracer la caractéristique de l'association en parallèle d'une diode Zener idéale de tension de seuil $U_s=\SI{0.1}{\volt}$, de tension de Zener $U_Z = \SI{0.7}{\volt}$ et d'un conducteur ohmique de résistance $R=\SI{1}{\kilo\ohm}$.
		\item Même question en remplaçant la diode par deux diodes Zener en série reliées par leurs bases
		\item Exprimer la caractéristiques de deux générateurs de tension $(e_1, R_1)$ et $(e_2, R_2)$ associées en parallèle, si les deux forces électromotrices sont fléchées dans le même sens. Conclusion ?
		\item Même question pour deux générateurs de courant $(\eta_1, G_1)$, $(\eta_2, G_2)$ associés en série.
	\end{enumerate}
\end{exercice}
\begin{exercice}[Association de dipôles et point de fonctionnement]
	\begin{enumerate}
		\item Tracer les caractéristiques des dipôles $AB$ et $MN$ représentés ci-dessous :
		\item Tracer la caractéristique $u=f(i)$ des dipôles $AB$ et $MN$ associés en parallèle, avec $A$ reliè à$N$ et $B$ à $M$.
		\item En déduire les coodonnées du point de fonctionnement de cette association sans connexion avec l'extérieur.
	\end{enumerate}
\end{exercice}
		
\begin{exercice}[Puissance reçue par un conducteur ohmique]
	On relie un générateur $(E, r)$ avec un conducteur ohmique de résistance $R$ variable.
	\begin{enumerate}
		\item Déterminer l'xpression de l'intensité $i$ dans le circuit
		\item Exprimer la puissance $P$ reçue par $R$ et tracer l'allure de $P=f(R)$. Quelles sont les expressions (avec $E$ et $r$) de la valeur maximale $P_M$ de $P$ et la valeur $R_M$ de $R$ correspondante ?
		\item Reprendre le même raisonnement avec un générateur modélisé par le modèle de Norton $(j, g)$.
	\end{enumerate}
\end{exercice}

\begin{exercice}[Montage potentiométrique]
La résistance de la partie du potentiomètre entre $A$ et $B$ est $kR$ avec $k$ variant entre 0 et 1.
	\begin{enumerate}
		\item Déterminer les éléments $e_T$ et $r_T$ du générateur de Thévenin situé à gauche ed $A$ et $B$ sur le schéma. Application numérique pour $R=\SI{1}{\kilo\ohm}$, $E=\SI{30}{\volt}$ et $k=0.3$.
		\item Déterminer l'expression de $U$ en fonction de $E$, $k$, $R$ et $R_C$. En posant $\alpha = \frac{R_C}{R}$ et $y= \frac{U}{E}$, exprimer $y$ en fonction de $k$ et $\alpha$. Application numérique pour $R_C=\SI{5}{\kilo\ohm}$.
		\item On note $U_0$ la valeur de $U$ lorsque $R_C$ est infinie et $y_0 = U_0/E$. Déterminer l'écart relatif $\epsilon = \frac{y_0-y}{y_0}$ en fonction de $k$ et $\alpha$. Montrer que $\epsilon$ passe par un maximum lorsque $k$ varie et $\alpha$ demeure constant. Tracer l'allure des courbes donnant $y$ en fonction de $k$ pour différents valeurs de $\alpha$.
		\item Pour quelle valeur de $R_C$ la puissance reçue par $R_C$ est elle maximale ?
	\end{enumerate}
\end{exercice}

\begin{exercice}[Diode idéale]
	Dans le circuit ci-contre
\end{exercice}
